\documentclass[14pt]{extarticle}

\usepackage[english]{babel}
\usepackage[utf8]{inputenc}
\usepackage{hyperref}
\usepackage{graphicx,eso-pic}
\usepackage{newtxtext}
\usepackage{setspace}
\usepackage{lipsum}
\usepackage{titlesec}
\usepackage{pdfpages}
\usepackage{indentfirst}
\usepackage[bottom=1.5cm,top=2.5cm,left=2cm,right=2cm]{geometry}

\hypersetup{
    colorlinks=true,
    linkcolor=black,
    filecolor=magenta,
    urlcolor=blue,
    pdftitle={Internship Report},
    pdfpagemode=FullScreen,
    }

\urlstyle{same}

\makeatletter

\setlength{\parskip}{1em}

\newcommand\frontmatter{
    \cleardoublepage
    \pagenumbering{roman}
}

\newcommand\mainmatter{
    \cleardoublepage
    \pagenumbering{arabic}
}

\newcommand\backmatter{
    \if @openright
        \cleardoublepage
    \else
        \clearpage
    \fi
}

\makeatother

\titleformat{\section}[block]{\Large\bfseries\filcenter}{}{1em}{}


\title{Python Development}
\author{}
\date{}

\begin{document}

\frontmatter

\newgeometry{bottom=2cm,top=2cm,left=1.5cm,right=1.5cm}
\addcontentsline{toc}{section}{Title}
\maketitle

\vspace{-5em}

\begin{center}
    \singlespacing
\textbf {A SUMMER TRAINING REPORT} \\
\emph {Submitted by} \\
\textbf {RISHABH ANAND} \\
\textbf {19BCS4525 }   \\

\vspace{1cm}
\onehalfspacing
\emph {in partial fulfilment of Summer training for the award of the degree} \\
\emph {of} \\

\vspace{1.5cm}
\singlespacing

BACHELOR OF ENGINEERING \\
IN \\
COMPUTER SCIENCE and ENGINEERING\\

\vspace{1cm}

\includegraphics[scale=0.25]{private/seal.png}

\vspace{1cm}
\singlespacing

APEX INSTITUTE OF TECHNOLOGY\\
CHANDIGARH UNIVERSITY, GHARUAN\\
Mohali, Punjab \\

\onehalfspacing
August, 2021

\end{center}
\restoregeometry

\newpage
\addcontentsline{toc}{section}{Certificate}
\includepdf[pages={1}]{private/certificate.pdf}

\newpage
\addcontentsline{toc}{section}{Acknowlgement}
\section*{Acknowlgement}
\par I thank everyone who were, in one way or another, instrumental in the completion of my internship at Kamtech Associates Pvt. Ltd.

\par First, I would like to thank Mr. Samir Dutta for taking me under his wing during my time at Kamtech and supervising me. I would also like to thank Mr. Saksham Gupta for giving me an oppourtunity to work at his company and his projects.

\par I also want to give my thanks to all the fellow developers and consultants as well as my fellow interns and teammatesat Kamtech, who were extremely helpful and supportive throughout my time there.

\par A special thanks to my parents who took care of me and provided me with all the necessary tools I required during my internship tenure.


\newpage
\addcontentsline{toc}{section}{List of Figures}
\listoffigures

\newpage
\addcontentsline{toc}{section}{ABSTRACT}
\section*{ABSTRACT}
\onehalfspacing

I worked on various project under my 13 week tenure as a Python Developer at Kamtech Associates Pvt. Ltd. like :

Built a website using \textbf{Django / Python} as back-end and \textbf{HTML 5 / CSS 3} as front-end. The site, \href{https://edubild.com/bidfast/}{BidFast}, first-of-its-kind Tendering Automation Tool, which automates the tedious process of creating technical proposals in response to procurement notices (RFP, EOI and RFQ) published by various procurement agencies (UNDP, UNGM, AfDB, TED, etc.)

The Django tool was based on a dynamic hierarchical model of the company where it was being deployed to. However, the basic structure / access remained the same. A central admin, who can add / remove as well as control the access of various department admins for tendering. The various depratment admins then control the file structure for the tender and create the CV, which is also dynamic. Then, the tool manages to assemble the whole document using \textbf{Jinja templates} and \textbf{PyDocs}.

Deployed the same website as an onsite tool on industrial servers for core companies like \textbf{Accenture} and \textbf{KPMG}.

Worked on web-scraping, using \textbf{Selenium / BeautifulSoup} and \textbf{Jupyter} and data collation scripts to generate an end to end yearly medical database for \textbf{Rajasthan Foundation}, which included records, for the total population, like their medical insurances and the grant given by the government for those who can't afford their bills.

Generated a database for \textbf{Edubild} website, a subset of Kamtech. The database consists of 10,000+ MCQs scraped from various websites like \href{https:\\www.sanfoundary.com}{Sanfoundary} and \href{https:\\www.tutorialspoint.com}{Tutorials Point}. The script successfully generated a database for all the MCQs with answer key which was then deployed to take tests of individuals on the Edubild site.

\setlength{\parskip}{0em}

\newpage
\addcontentsline{toc}{section}{CONTENTS}
\begin{center}
    \tableofcontents
\end{center}

\mainmatter

\setlength{\parskip}{1em}

\newpage
\section{INTRODUCTION}

\par \textbf{Kamtech Associates Private Limited} is a a technology based consultancy and solutions company working across industry and business domains operating in emerging and mature economies like Africa, Europe, Middle East and Asia and a two-time National Award winner by Government of India.

\par A subset of Kamtech Associates is \textbf{EduBild}. Among the many projects that were undergoing in the company at that moment, I got to work mainly on BidFast, which is the simplest tool for your tendering needs.

\par \textbf{BidFast} has the capability to automate the generation of complete tenders for various types of proposals like RFPs,RFQs, RFIs, EOIs, RFx along with business proposals for your company. It has a one-time content creation and storage management system that ensures minimum human interaction and safety for your precious data on a fundamental level.

\par The one of a kind content management system that I personally worked on ensures effortless document creation. It ensures that there is absolutely no more mindless copying and pasting, no more outdated documents/files and no more room for human errors what-so-ever. BidFast offers well structured content management for all your tender documents.

\par Apart from BidFast I also worked on some small web-scraping assignments which included some advanced web-crawling and automated dataset generation for EduBild's experienships tests and selections. The dataset generqated was directly uploaded to their server and the test results and selections were taken for more than 500 participants/applicants.

\newpage
\section{THEORY}
\par During my tenure, I used a number of python libraries like :
\begin{itemize}
    \item Flask
     \begin{itemize}
        \item \par Flask is a micro web framework written in Python. It is classified as a microframework because it does not require particular tools or libraries. It has no database abstraction layer, form validation, or any other components where pre-existing third-party libraries provide common functions
        \item \par The tool BidFast was first create dusing Flask. It was serving the purpose but then the whole system was switched to Django, mainly due to Flask's scalability issues.
    \end{itemize}
    \item Django
    \begin{itemize}
        \item \par Django is a Python-based free and open-source web framework that follows the model–template–views architectural pattern.
        \item \par The tool BidFast was re-created using BidFast with better UI/UX, faster workflow and even more versatility.
        \item \par After switching to Django, we had to add support for databases and make a better frontend for the overall project. \textbf{SQLAlchecmy}, an open-source SQL toolkit and object-relational mapper for the Python programming language released under the MIT Licens,  was used for creating/managing the database layer. It provides agood wrapped environment for SQLs inside python. The database framework used was \textbf{sqlite} because of it's simplicity and speed.
        \item \par Apart from the backend, the front-end was also redesigned keeping with the newer framework with help nof \textbf{BootStrap},Bootstrap is a free and open-source CSS framework directed at responsive, mobile-first front-end web development, which contains CSS- and JavaScript-based design templates for typography, forms, buttons, navigation, and other interface components, which offers templated and well documented html/css snippets for faster website creation.
    \end{itemize}
    \item Selenium
     \begin{itemize}
        \item \par Selenium is an open-source automated testing framework for web applications. Selenium provides a playback tool for authoring functional tests without the need to learn a test scripting language.
        \item \par The usage of Selenium is standarized for web automation at this point in the market. Selenium is basically used to automate the testing across various web browsers. It supports various browsers like Chrome, Mozilla, Firefox, Safari, and IE.
    \end{itemize}
    \item BeautifulSoup
    \begin{itemize}
        \item Beautiful Soup is a Python package for parsing HTML and XML documents. It creates a parse tree for parsed pages that can be used to extract data from HTML, which is useful for web scraping.
        \item It basically parses a HTML or XML web-page and generates a minimal text only content tree for the web-pages it's being used on for data extraction and string conversion.
    \end{itemize}
    \item Others
    \begin{itemize}
        \item Pillow - In automatic CV generation, Pillow was used at times for images processing and page manipulation.
        \item Jinja - Jinja is a document-tagging and temporary placeholder generator package that switches tags in document with the specified string provided at runtime.
        \item PyDocs - Python wrappper for Docx generation and manipulation. Works with most MS-Office formats and a few Open-Source Document formats.
    \end{itemize}
\end{itemize}

\newpage
\section{METHODOLOGY ADOPTED}
\subsection{BidFast}

\begin{figure}[hbp]
    \centering
    \caption{BidFast}
    \includegraphics[width=0.95\textwidth]{private/BF.png}
\end{figure}

BidFast is built on Django and uses sqlite for the backend. Users can generate a CV Database using the intuitive generator that I made. The whole system uses position based access, i.e., if a person is admin for a particular department he/she cannot see/alter the CVs or any data for that matter, and only access his/her files. The CV data stored in the database is fetched by the countless templates created by the companies or the basic ones that were preconfigured by our team. We use PyDocs to first generate the doc, mixed with Jinja templates to fill the data obtained from the backend.

To generate the position based access, we used different Django views. And a custom function that ensured the protction of data.

\subsection{EduBild}

EduBild, as mentioned above, provides a platform for testing and assessing the skills of candidates that apply for an experienship. The site has a database of 10,000+ questions curated from different sites and resources. The script was develpoed solely by me and was able to automatically crawl, parse and understand the Q/As in MCQ formast using Natural Language Processing and Machine Learning. The program was also able to store all the data directly to the databaseusing APIs.

\begin{figure}[hbp]
    \centering
    \caption{EduBild}
    \includegraphics[width=0.95\textwidth]{private/EB.png}
\end{figure}


\newpage
\subsection{Scraping}

Apart from these two main projects, I also worked on some side-projects small projects for the Rajasthan Government through Kamtech. I worked on generating a program to parse and curate the data from various medical records and databases.

\begin{figure}[hbp]
    \centering
    \caption{Kamtech}
    \includegraphics[width=0.95\textwidth]{private/KA.png}
\end{figure}


\newpage
\section{CONCLUSIONS}

During the three months tenure, that I served as a Python Developer at Kamtech, I learned a number of things that will help me in my career and my life. I learned how to handle a team when I was working on BidFast as I was maintaning the git repository and the trello boards for the overall development while also coding the software. I helped my understand how to manage people and get work done.

I learned making sites woth Django. I learned how to work with views and forms. I learned control based access on data. I learned basics of Web Development while building the website.

While working on EduBild I learned advanced web-scraping and data retention. I learned working with SQL and advanced Natural Language Processing. I learned the importance of experience in teh field of IT. I learned the fundamentals of text processing.

While working on the side projects I learned the very important skill of time management. I learned how to work with government databases and the level of security and clearance needed to access the data stored there.

Apart from these, I learned skills like communication and group discussion which are equally important as technical skills if you have to survive inside corporate. I learned why it is important to present my point of view in certain situation and how to do that. I learned why it is important to participate in group activities and how to work with people who want to contribute to the same cause.

Moreover, I feel more confident now that I have finished an internship in the industry. I feel that I have the skills and knowledge needed to get inside the corporate world. I feel I can succeed.

\newpage
\addcontentsline{toc}{section}{REFERENCES}
\section*{REFERENCES}
\begin{enumerate}
    \item \href{https://www.kamtech.in/}{Kamtech}
    \item \href{https://www.edubild.com/}{EduBild}
    \item \href{https://edubild.com/bidfast/}{BidFast}
\end{enumerate}

\end{document}