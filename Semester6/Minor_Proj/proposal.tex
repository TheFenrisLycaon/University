\documentclass{article}

\usepackage[english]{babel}
\usepackage[utf8]{inputenc}
\usepackage{graphicx,eso-pic}
\usepackage{newtxtext}
\usepackage{array}
\usepackage{lipsum}
\usepackage{indentfirst}
\usepackage{setspace}
\usepackage[bottom=1.5cm,top=1.5cm,left=2cm,right=1.5cm]{geometry}

\pagenumbering{gobble}
\setlength{\parskip}{1em}

\title{Project Proposal}
\author{}
\date{}


\begin{document}

\includegraphics[width=0.99\textwidth]{private/Header.jpeg}
\vspace{3em}

\section{Project Title}

\Large{AfterShock}

\section{Project Scope}

In a world, divided by fear, of losing your loved ones, of losing your loved belongings,of losing your life, we hope to come up with a solution that should keep you and your dreams safe. Because that's what EarthQuake's take away... Even after the major tremor, what hurts more is the AfterShocks that follow. These are produced by the stress that was caused by the earthquake.

This project gives us a second chance at saving lives by using Artificial Intelligence to determine where the next tremor is going to be. So that you can move, and get to a safer place. Methods like Columnb's Stress Criterion are being used in current times to explain the spatial distributions of AfterShocks, but as the advent of science \& technology is improving, we hope to introduce Machine Learning models that can find an undiscovered pattern which will be helpful in predicting the fair locations of AfterShocks.

Once we have our predictions, it is very important to display them in a good manner so that Uncle Bob can understand them and move himself to safety. We have created a React web-app just for this purpose so that it is easily acessible to people and move them from harm's way. Thereby, reducing the damage to both people and resources, thus, making this world a better place.

\section{Requirements}

\begin{tabular}{ll}
\textbf{Requirements} & \textbf{Description} \\
Python & To Predict the AfterShocks \\
React & To display the predictions and deliever teh information \\
Node & To store the data from python models and talk to the Front-End for display.\\
\end{tabular}

\section*{Student Details}

\begin{center}
\begin{tabular}{ |m{0.25\textwidth}|m{6em}|m{7em}| }
    \hline
    \textbf{Student Name} & \textbf{UID} & \textbf{Signature} \\
    Rishabh Anand & 19BCS4525 & \begin{center}\raisebox{-0.5em}{\includegraphics[scale=.45]{private/rSign.jpeg}}\end{center}\\
    Abhishek Singh & 19BCS4508 &\begin{center} \raisebox{-0.5em}{\includegraphics[scale=0.06]{private/aSign.jpeg}}\end{center}\\
    Shefali Yadav & 19BCS4524 & \begin{center}\raisebox{-0.5em}{\includegraphics[scale=0.13]{private/sSign.jpeg}}\end{center}\\
    \hline
\end{tabular}
\end{center}

\section*{Approval and Authority to Proceed}

We approve the project as described above,and authorize the team to proceed.


\begin{center}
\begin{tabular}{ |p{9em}|p{9em}|p{9em}| }
    \hline
    \begin{center}\textbf{Name}\end{center} & \begin{center}\textbf{Title}\end{center} & \begin{center}\textbf{Signature}\end{center} \\
    \hline
    \begin{center}{Nikhil Aggarwal}\end{center} & \begin{center}{Supervisor}\end{center}&\\
    \hline
\end{tabular}
\end{center}

\end{document}