\documentclass[14pt]{extarticle}

\usepackage[english]{babel}
\usepackage[utf8]{inputenc}
\usepackage{graphicx,eso-pic}
\usepackage{tgschola}
\usepackage{listings}
\usepackage[nodayofweek]{datetime}
\usepackage[bottom=2.25cm,top=3.75cm,left=1.5cm,right=1.5cm]{geometry}
\usepackage{xcolor}

\fontfamily{qcs}
\pagenumbering{gobble}

\lstdefinestyle{mystyle}{
    basicstyle=\normalsize,       
    breaklines=true,                                 
    keepspaces=true,                                    
    numbersep=5pt,                  
    showspaces=false,                
    showstringspaces=false,
    showtabs=false,                  
    tabsize=2
}

\lstset{style=mystyle}

% Add header and footer images
\AddToShipoutPictureBG{%
  \AtPageUpperLeft{%
    \raisebox{-\height}{%
    \includegraphics[width=\paperwidth]{private/header.jpeg}}
  }
  \AtPageLowerLeft{%
    \makebox{%
    \includegraphics[width=\paperwidth]{private/footer.jpeg}}
  }
}

% Set title
\title{%
    \textbf{
    \vspace{-3em} \\ 
    \Large Experiment Number \\
    \vspace{-4em}
    }
}

% Remove author and Date
\author{}
\date{}

\begin{document}

\maketitle % Add title of doc

% Add the name and details
\section*{}
    \begin{tabular}{ llp{2cm}ll } 
        Name :: & Rishabh Anand & & UID :: & 19BCS4525  \\ 
        Branch :: & CSE - IoT & & Sec/Grp :: & 1/A \\ 
        Semester :: & 5\textsuperscript{th} & & Date :: & \shortdate{\today} \\
        Subject :: & Adv Programming Lab & & CODE :: & CSP-347  \\ 
        % Subject :: & Embedded System Lab & & CODE :: & CSD-333  \\ 
        % Subject :: & DIOT Lab & & CODE :: & CSD-337  \\ 
        % Subject :: & WSN Lab & & CODE :: & CSD-331  \\ 
    \end{tabular}
    
\vspace{1em}

\section*{\normalsize 1. Aim :}
    
Implement a class template Gradebook with two nontype template parameters that represent the number of students and the number of tests . Test multiple instances of the GradeBook class template with different numbers of students and exams.

\section*{\normalsize 2. Task :}
    
Make a template class to compute two take parameters. And then, test it with different data

\section*{\normalsize 3. Algorithm :}
    
    \begin{enumerate}
        \item Make a template function
        \item Compute maximum
        \item Test function with three data types
        \begin{enumerate}
            \item int
            \item float
            \item char
        \end{enumerate}
    \end{enumerate}

% Auto add code and output
\section*{\normalsize 4. Steps / Source Code :}

    \lstinputlisting{temp/code.cpp} 

    \vspace{1em}

\section*{\normalsize 5. Observations :}

Following code executes without any error and code duplication.

\section*{\normalsize 6. Result :}

\includegraphics{temp/output.jpeg}

\section*{\normalsize Learning Outcomes :}
    
    \begin{itemize}
      \item C++ templates
      \item STL library
    \end{itemize}

    \section*{}
\begin{center}
\begin{tabular}{ |p{2.5cm}|p{4cm}|p{5cm}|p{5cm}|} 
 \hline
 S. No. & Parameters & Marks Obtained & Maximum Marks \\
 \hline
 1.&&&\\
 \hline
 2.&&&\\
 \hline
 3.&&&\\
 \hline
 &&&\\
 &&&\\
 \hline
\end{tabular}
\end{center}

\end{document}